\documentclass[a4paper]{article}
\usepackage[ngerman]{babel}
\usepackage{amsthm}
\usepackage{amsmath}
\usepackage{amsfonts}
\usepackage{parskip}
\usepackage{graphicx}
\usepackage{color}
\usepackage{listings}
\usepackage{microtype}

\usepackage{color}
\usepackage[colorlinks, linkcolor = black, citecolor = black, filecolor = black, urlcolor = blue]{hyperref} 

%\PassOptionsToPackage{bookmarksopen,
%pdfstartview={Fit},
%colorlinks=true,
%linkcolor=blue,
%urlcolor=blue,
%citecolor=blue,
%plainpages=false,
%pdfpagelabels}{hyperref}
\makeatletter
\g@addto@macro{\maketitle}{\def\author#1{\def\@author{#1}}}
\newcommand*{\extendsectlevel}[1]{%
  \expandafter\newcommand\expandafter*\csname saved@#1\endcsname{}%
  \expandafter\let\csname saved@#1\expandafter\endcsname\csname #1\endcsname
  \expandafter\renewcommand\expandafter*\csname #1\endcsname{%
    \expandafter\let\csname author@#1\endcsname\@author
    \@ifstar
      {\csname star@#1\endcsname}%
      {\@dblarg{\csname opt@#1\endcsname}}%
  }%
  \expandafter\newcommand\expandafter*\csname star@#1\endcsname[1]{%
    \csname saved@#1\endcsname*{##1%
      \expandafter\ifx\csname author@#1\endcsname\@empty\else
        \hfill\linebreak{\normalsize
          \textmd{\textit{\csname author@#1\endcsname}}}%
      \fi
    }%
  }%
  \expandafter\newcommand\expandafter*\csname opt@#1\endcsname[2][]{%
    \csname saved@#1\endcsname[{##1%
      \expandafter\ifx\csname author@#1\endcsname\@empty\else
%        \enskip\textmd{\textit{(\csname author@#1\endcsname)}}%
% >>> UN/COMMENT the above line to enable/disable author names in the TOC
      \fi
    }]{##2%
      \expandafter\ifx\csname author@#1\endcsname\@empty\else
            \\ 
            \normalsize
            \textmd{\textit{\csname author@#1\endcsname}}
      \fi
    }
  }
}
\extendsectlevel{section}
\extendsectlevel{subsection}
\makeatother

\definecolor{mygreen}{rgb}{0,0.6,0}
\definecolor{mygray}{rgb}{0.5,0.5,0.5}
\definecolor{mymauve}{rgb}{0.58,0,0.82}

\lstset{ 
  backgroundcolor=\color{white},
  basicstyle=\footnotesize,
  breakatwhitespace=false,
  breaklines=true,
  captionpos=b,
  commentstyle=\color{mygreen},
  extendedchars=true,
  frame=single,
  keepspaces=false,
  keywordstyle=\color{blue},
  language=Java,
  numbers=left,
  numbersep=5pt,
  numberstyle=\color{mygray},
  rulecolor=\color{black},
  showspaces=false,
  showstringspaces=false,
  showtabs=false,
  stepnumber=2,
  stringstyle=\color{mymauve},
  tabsize=2,
  title=\lstname
}

\pagestyle{headings}

\newtheorem{satz}{Satz}
\theoremstyle{definition}
\newtheorem{definition}{Definition}
\setcounter{section}{-1}
\title{Dokumentation}
\author{
    Matthias Vonend
    \and
    Jan Grübener
    \and
    Patrick Mischka
    \and
    Michael Angermeier
    \and
    Troy Keßler
    \and
    Aaron Schweig
}

\begin{document}

    \maketitle

    \begin{center}
      \href{https://github.com/aaronschweig/vs-chat/}{GitHub}
      ·
      \href{https://youtu.be/ZKApXYr4j58}{Video}
    \end{center}

    \tableofcontents
    \clearpage
    %\vspace{0.2cm}

    \author{}
    \section{Quick Start Guide}
    Zunächst sollte sichergestellt werden, dass als Java-Version 11 eingestellt ist. Das Projekt kann als Maven-Projekt importiert werden.
    Danach kann die erste Node gestartet werden. Dafür muss eine Instanz der Klasse Server-Bootstrapper erzeugt werden
    und wenn es erwünscht ist, ein anderer Port (standardmäßig 9876) eingestellt werden. Außerdem können mit der Klasse \textit{NodeConfig} weitere Nodes angegeben werden, zu dehnen eine Verbindung gehalten werden soll.
    Wurde mindestens eine Node erfolgreich hochgefahren, können beliebig viele Clients gestartet werden. Bei der Klasse Client
    können noch Programmargumente in der Form \textit{java xy.jar node1.net:1234 node2.de:9877} mitgegeben werden. Stehen in den Programmargumenten die Adressen der Nodes, werden
    diese ausgewählt, oder es wird standardmäßig \textit{localhost} ausgewählt.
    Sind 2 Clients online, kann ein Chat erstellt werden.
    \author{}
    \section{Basisanforderungen}
        % !TEX root = ./docs.tex

\begin{figure}[h]
    \centering
    \includegraphics[width=\textwidth]{architecture.png}
    
    \caption{Architektur}
\end{figure}

\author{Matthias Vonend}
\subsection{Chatfunktionalität}\label{Chatfunktionalitaet}
Die Anwendung ist mit einem Thin Client aufgebaut. Damit ein Chat ablaufen kann, muss zunächst eine Verbindung zu einem Server aufgebaut werden.
Dazu wählt der Client zunächst einen zufälligen Server aus und versucht sich zu verbinden.
Wurde eine Verbindung erfolgreich aufgebaut, kann sich der Nutzer mit seinem Nutzernamen und seinem Passwort anmelden.
Sobald der Nutzer angemeldet ist, sendet der Server ihm alle benötigten Informationen inklusive der verpassten
Nachrichten zu. Der Server bergibt jeder eintreffenden Nachricht einen Timestamp, um zu dokumentieren, wann sie erstmalig eingetroffen ist.
Anhand des Timestamps werden die Nachrichten sortiert, damit der Client die korrekte Reihenfolge der Nachrichten darstellen kann.

Wenn ein Client eine Nachricht versenden möchte, wird das Nachrichtenpaket an die Node gesendet, mit der er verbunden ist.
Die Node kümmert sich im Hintergrund darum, die Nachricht an den Zielclient zuzustellen. 
Da alle Nachrichten aus Konsistenzgründen an alle Nodes verteilt werden müssen,
brauchen die Nodes keine Information über die Clients anderer Nodes. Im Falle einer solchen Anforderung
(z.\,B. Abfrage, ob ein anderer Nutzer aktiv ist) könnte ein Protokoll ähnlich zu Routing-Tabellen implementiert werden, um die zusätzliche Funktionalität bereitzustellen.
Empfängt eine Node eine Nachricht, egal ob von einem Client oder von einer anderen Node,
wird überprüft, ob die Nachricht für einen ihr bekannten Client bestimmt war. Wird ein Client gefunden, sendet die Node die Nachricht an den Zielclient.

\author{Jan Grübener, Troy Keßler, Patrick Mischka, Michael Angermeier}
\subsection{Clientfunktionalitäten}
Nach einer erfolgreichen Anmeldung kann der Nutzer zwischen verschiedenen Funktionen auswählen:
% \begin{itemize}
%     \item /help
%     \item /chats
%     \item /contacts
%     \item /createchat
%     \item /openchat
%     \item /exit
% \end{itemize}

\subsubsection*{/help:}
Diese Funktion gibt dem Nutzer einen Überblick über alle möglichen Funktionen, die er aufrufen kann.
Alle Funktionen sind kurz beschrieben, sodass der Nutzer einen Überblick über die Funktionen erhält.

\subsubsection*{/chats:}
Bei einem Aufruf dieser Funktion werden alle Chats, die für den Nutzer zugänglich sind, angezeigt.
%Dafür werden zunächst alle Chats durch die Methode getChats aus der API in einem Set aus Chats gespeichert.
Wenn Chats für den Nutzer verfügbar sind, werden diese in einer Übersicht mit Chatname und teilnehmenden Nutzen dargestellt.
Sind noch keine Chats vorhanden, wird darauf hingewiesen.

\subsubsection*{/contacts:}
Wird \textit{/contacts} aufgerufen, werden alle registrierten Nutzer angezeigt.

\subsubsection*{/createchat:}
Diese Funktion beginnt mit einer Aufforderung an den Nutzer, einen Chatnamen einzugeben.
Danach wird die Anzahl der Teilnehmer für den Chat erfragt, wobei mindestens ein Teilnehmer im Chat enthalten sein muss.
Im nächsten Schritt müssen alle Nutzernamen der Teilnehmer eingetragen werden. 
Jeder Nutzername wird auf seine Gültigkeit geprüft. Ist er ungültig, so wird eine Information darüber ausgegeben (//TODO Määh). 
%Hierfür wird jeder einzelne Nutzername überprüft und
% im Falle eines ungültigen Nutzernamens, wird der Nutzer durch eine Meldung darauf aufmerksam gemacht.
Wurden alle drei Attribute (Chatname, Teilnehmeranzahl, Nutzername der Teilnehmer) erfolgreich eingegeben,
wird ein neuer Chat erstellt.

\subsubsection*{/openchat:}
Will der Nutzer einen Chat öffnen, so muss er zuerst den Chatnamen eingeben. Ist der Chat vorhanden, so wird er geöffnet.
Kann der Chat nicht geöffnet werden, so wird eine Meldung für den Nutzer ausgegeben.
Am Anfang eines Chats wird immer darauf hingewiesen, wie der Chat
verlassen werden kann. Danach werden alle Nachrichten, die in diesem Chat bereits geschrieben wurden, geladen.
Anschließend kann der Nutzer Nachrichten versenden und empfangen.

\subsubsection*{/exit:}
Mithilfe dieser Funktion wird der Nutzer abgemeldet und Client beendet.
\clearpage

\author{Matthias Vonend, Aaron Schweig, Troy Keßler}
\subsection{Fehlerbehandlung}
Um Fehler und Datenverluste zu vermeiden, ist in dem Chatsystem sichergestellt, dass immer mindestens zwei Server (Nodes) \textbf{alle}
Informationen besitzen. So kann während eines Nodeausfalls gewährleistet werden, dass eine Andere alle Aufgaben übernehmen kann.

%Aus den Anforderungen geht hervor, dass es mindestens zwei Server geben muss, die sämtliche Informationen des
%Chatsystems besitzen müssen. Bricht eine Node zusammen, muss eine andere Node dessen Aufgabe übernehmen.
\subsubsection{Client}
Nachdem der Client eingeloggt ist, wartet er kontinuierlich auf neue Pakete vom Server. Stürzt eine Node ab, oder verliert der Client
die Netzwerkverbindung, so versucht er sich neu zu verbinden. Dabei wird eine neue, zufällige Node ausgewählt.
Kann der Client erfolgreich eine neue Verbindung aufbauen, meldet sich der Client mit den bestehenden Zugangsdaten an der Node an.
Ist diese nicht verfügbar, so wird so lange eine neue Node ausgewählt, bis eine Verbindung zustande kommt.
Im Regelfall findet der beschriebene Reconnect-Vorgang im Hintergrund statt.

//TODO * (zus. in der Erweiterung: Teilnehmer ist nicht online)

\subsubsection{Server}
Serverseitig können verschiedene Fehler auftreten. Viele Fehler werden bereits durch das TCP-Protokoll und Java selbst verhindert 
(z.\,B. Mehrfachzustellung, fehlerhafte Übermittlung, \dots). Dennoch können grundsätzlich die folgenden Fehlerfälle eintreten:

\subsubsection*{Nachricht des Clients wird nicht korrekt gesendet/empfangen}
In diesem Fall muss der Server davon ausgehen, dass die Verbindung zusammengebrochen ist. Diese wird im Anschluss vom Server beendet
und wie bereits oben beschrieben versucht der Client erneut eine Verbindung aufzubauen.
Ist eine neue Verbindung wiederhergestellt, werden alle für den Client relevanten Nachrichten zu diesem übertragen 
und der Client muss fehlende Informationen erneut an den Server übertragen.

\subsubsection*{Nachrichten einer Nachbarnode werden nicht korrekt empfangen/gesendet}
Wie bereits im Kapitel \ref{Chatfunktionalitaet} beschrieben, tauschen Nodes alle Nachrichten untereinander aus. Innerhalb des Chatsystems können mehrere
Netzwerktopologien realisiert werden. Die ausfallsicherste Variante stellt dabei eine Mesh-Struktur dar. 
Diese ermöglicht es, auch bei einem Ausfall mehrerer Nodes, alle Nachrichten mit allen bekannten Nachbarnodes zu synchronisieren. Netzwerkpartitionierungen sind damit
deutlich schwerer zu erreichen und auch mehrerer Nodeausfälle können mithilfe der Mesh-Topologie kompensiert werden. Ebenfalls wird durch die Wahl der
Write-All-Available Replikationsstrategie sichergestellt, dass alle bekannten Nachbarnodes komplett synchronisiert sind.
\\
Tritt ein Fehler in der Verbindung zwischen verschiedenen Nodes auf, so muss die Node ähnlich wie bei einem Fehler im Client davon ausgehen, 
dass die Verbindung zusammengebrochen ist. Allerdings sind die Nodes hier selbst für eine Fehlerbehandlung zuständig und versucht eine neue
Verbindung aufzubauen. Um sicherzustellen, dass keine konkurrienden Schreibzugriffe auf den OutputStream stattfinden,
werden Nachrichten in einer Queue aufbewahrt bevor sie an andere Nodes übertragen werden.
\\
Werden die Nodes getrennt sind die Clients immer noch in der Lage Nachrichten an ihre jeweiligen Nodes zu schicken.
Sobald eine Verbindung wieder aufgebaut wurde, synchronisieren sich die Nodes, um einen vollständigen Informationsstand wiederherzustellen.
Sind keine Nachrichten zu senden, hat die Node keine Möglichkeit festzustellen, ob eine Verbindung noch existiert. 
Zu diesem Zweck existiert ein Heartbeat, der periodisch die Nachbarnodes anpingt und so prüft, ob die Verbindung noch existiert.
Jede Node prüft dabei, 
ob sie alle Informationen des empfangenen Warehouses bereits besitzt und fügt geg. neue Informationen hinzu. 
Sofern die Node neue Informationen erhalten hat, broadcastet diese ihren neuen Stand an alle benachbarten Nodes,
um diese auch auf den neuesten Stand zu bringen.




    \author{}
    \section{Erweiterungen}
        % !TEX root = ./docs.tex

\subsection{Grafische Oberfläche bei den Nutzern}

\subsection{Verwendung von Emojis}

\subsection{Mehrere Chatverläufe pro Nutzer}

\subsection{Persistentes Speichern der Chatverläufe}

\subsection{Verschlüsselte serverseitige Speicherung der Chats}

\subsection{Gruppenchats}

\subsection{Verschlüsselte Übertragung der Chat-Nachrichten}

%\subsection{3 replizierte Server mit Majority Consensus Strategie}
% Gibt in unserem Ansatz keinen Sinn


    \clearpage
    \author{}
    \section{Codewalkthrough}
        % !TEX root = ./docs.tex

Bei der Implementierung wird Java 11 eingesetzt.
\subsection{Server}
Einstiegspunkt des Servers ist der Serverbootstrapper. Dieser erstellt einen neuen Thread mit einem neuen Server. 
\lstinputlisting[linerange={7-11}, firstnumber=7]{../src/main/java/vs/chat/server/ServerBootstrapper.java}
Der Server erstellt die Listener, die die zu empfangenen Pakete behandeln werden. Anschließend werden die Filter erstellt, die bestimmen, ob ein Paket gehandelt oder ignoriert werden soll (z.\,B. bei rekursiven Broadcasts).
\lstinputlisting[linerange={64-80}, firstnumber=64]{../src/main/java/vs/chat/server/Server.java}

Die Listener und die Filter werden in einen ServerContext gepackt, der mit allen Threads geteilt wird.
\clearpage

\begin{figure}[h]
    \centering
    \includegraphics[width=\textwidth]{VS-Server-Context.png}
    
    \caption{Server-Context Aufbau}
\end{figure}


Sobald der ServerContext instanziiert wird, wird das Warehouse geladen. 
Zunächst wird versucht die Safe-Datei zu laden. Scheitert das Laden, wird das Warehouse leer instanziiert.

\lstinputlisting[linerange={53-64}, firstnumber=53]{../src/main/java/vs/chat/server/warehouse/Warehouse.java}


Das Warehouse hält sämtliche Daten, die persistiert werden müssen (z.\,B. Messages, Chats, Users). Der ServerContext erstellt außerdem den Persister. Der Persister ist ein Thread, der in regelmäßigen Abständen das Warehouse speichert.
\lstinputlisting[linerange={17-31}, firstnumber=17]{../src/main/java/vs/chat/server/persistence/Persister.java}
\lstinputlisting[linerange={66-75}, firstnumber=66]{../src/main/java/vs/chat/server/warehouse/Warehouse.java}
Alle Entitäten und Pakete haben eine UUID (Universally Unique Identifier) um diese zu unterscheiden. Verweise auf andere Entitäten (z.\,B. ein Chat hat mehrere Nutzer) werden ähnlich zu Fremdschlüsseln in relationalen Datenbanken umgesetzt. Die Entität speichert nur die UUID der Verknüpfung und nicht direkt die Information. Dies erlaubt eine feinere Synchronisation und reduziert mögliche Konfliktsituationen zwischen verschiedenen Nodes. Eingesetzt werden UUIDv4, die pseudozufällig erstellt werden. Dadurch sind zwar theoretisch Konflikte möglich, jedoch in Praxis sehr unwahrscheinlich.

Außerdem wird der Broadcaster erstellt, der die Verbindungen zu anderen Nodes hält und empfangene Nachrichten an diese verteilt.

\lstinputlisting[linerange={14-27}, firstnumber=14]{../src/main/java/vs/chat/server/node/NodeBroadcaster.java}


Weitere Variablen sind \textit{isCloseRequested}, die die 'endlos' Schleifen aller Threads steuert und \textit{connections}, welche alle Verbindungen zu Clients, die direkt zu dieser Node verbunden sind, hält.

Der Nodebroadcaster erstellt bei Instanziierung je Node einen eigenen Thread, der sich um das Senden und das neu Verbinden kümmert.
Nachrichten, die an eine Node gesendet werden sollen werden sollen werden vom Broadcaster in die Queue geschrieben und die NodeConnection wird durch einen Semaphor aufgeweckt. 
Die NodeConnection versucht eine Nachricht zu senden. Scheitert das Senden wird von einer Verbindungstrennung ausgegangen und die Verbindung wird neu aufgebaut.

\lstinputlisting[linerange={37-50}, firstnumber=37]{../src/main/java/vs/chat/server/node/NodeConnection.java}
Zusätzlich besitzt die NodeConnection jeweils einen HeartBeat-Thread. Dieser Thread sendet regelmäßig einen Ping, um zu testen, ob die Verbindung noch steht.
\lstinputlisting[linerange={16-25}, firstnumber=16]{../src/main/java/vs/chat/server/node/NodeHeartBeatThread.java}

Nachdem nun alle Initialisierungsvorgänge abgeschlossen sind, kann der ServerSocket erstellt und Clients akzeptiert werden.
Der Hauptserver-Thread ist dabei nur zuständig neue Verbindungen entgegenzunehmen. Für jede Verbindung wird ein ConnectionHandler-Thread erstellt, der sämtliche Nachrichten des Clients verarbeitet.
\lstinputlisting[linerange={41-60}, firstnumber=41]{../src/main/java/vs/chat/server/Server.java}

Nachrichten zwischen Servern und Clients werden als Pakete ausgetauscht. Der Handler versucht dabei ein Paket vom Client zu lesen. Die Filter prüfen nun, ob das Paket gehandelt werden darf (und nach dem Verarbeiten dessen werden diese aktualisiert).
\lstinputlisting[linerange={47-53}, firstnumber=47]{../src/main/java/vs/chat/server/ConnectionHandler.java}
Anschließend werden die passenden Listener gesucht und diese mit dem Paket aufgerufen.
\lstinputlisting[linerange={61-86}, firstnumber=61]{../src/main/java/vs/chat/server/ConnectionHandler.java}



Filter:
\begin{itemize}
    \item PacketIdFilter\\
        Der PacketId-Filter testet, ob ein Paket mit der Id bereits gesehen wurde. Nur wenn die Id neu ist darf das Packet gehandelt werden, um rekursive Broadcasts zu vermeiden. Bereits gesehene Pakete werden im Warehouse mitgespeichert. Im seltenen Fall, dass die Node genau zwischen den Listenern und dem Aktualisieren der Filter abstürzt kann es vorkommen, dass die gespeicherten Paket-ids nicht konsistent zum Nutzdatenbestand sind. Hier könnte ein Transaktionsprotokoll implementiert werden. Da aber die Wahrscheinlichkeit dieses Fehlers äußerst gering ist wird hier darauf verzichtet.
        \lstinputlisting[linerange={9-17}, firstnumber=9]{../src/main/java/vs/chat/server/filter/PacketIdFilter.java}
\end{itemize}



Listener:
\begin{itemize}
    \item BaseEntityBroadcastListener\\
        Der Listener behandelt BaseEntityBroadcastPackete, die ausgestrahlt werden, sobald ein neuer Nutzer, ein neuer Chat oder eine neue Nachricht erstellt wird. Die empfangene Entität wird in das Warehouse aufgenommen und weiter gesendet, falls es dieser Node neu war.
        \lstinputlisting[linerange={22-28}, firstnumber=22]{../src/main/java/vs/chat/server/listener/BaseEntityBroadcastListener.java}
        Nachdem ein Chat erstellt wurde müssen die Clients, die an diesem Chat teilnehmen informiert werden. Da jede Node nur die direkt zu ihr verbundenen Clients kennt, muss jede Node prüfen ob sie einen teilnehmenden Client kennt und diesen informieren. Ähnliches gilt für neue Nutzer.
        \lstinputlisting[linerange={33-50}, firstnumber=33]{../src/main/java/vs/chat/server/listener/BaseEntityBroadcastListener.java}
    \item CreateChatListener\\
        Dieser Listener erstellt Chats anhand von einem CreateChatPacket. Sofern das Paket von einem Nutzer kommt, wird ein neuer Chat mit allen Teilnehmern und dem Absender erstellt und weiter verteilt.
        \lstinputlisting[linerange={23-41}, firstnumber=23]{../src/main/java/vs/chat/server/listener/CreateChatListener.java}
    \item GetMessagesListener\\
        Mithilfe eines GetMessagePackets können alle Nachrichten abgefragt werden, die in einem Chat gesendet wurden.
        \lstinputlisting[linerange={20-32}, firstnumber=20]{../src/main/java/vs/chat/server/listener/GetMessagesListener.java}
    \item KeyExchangeListener\\
        Dieser Listener leitet KeyExchangePackete von einem Client an andere Clients weiter (gegebenenfalls über andere Nodes).
        \lstinputlisting[linerange={15-27}, firstnumber=15]{../src/main/java/vs/chat/server/listener/KeyExchangeListener.java}
    \item LoginListener\\
        Der LoginListener kümmert sich um die Authentifizierung eines Clients. Er prüft, ob ein Nutzer besteht und falls ja wird das Passwort geprüft.
        Außerdem wird die Information der Connection gesetzt, zu welchem Client sie verbunden ist, um ein gezieltes Senden zu ermöglichen (wie z.\,B. beim KeyExchange oder bei Messages).
        \lstinputlisting[linerange={25-34}, firstnumber=25]{../src/main/java/vs/chat/server/listener/LoginListener.java}
        Existiert noch kein Nutzer, wird ein passender Nutzer erstellt.
        \lstinputlisting[linerange={37-56}, firstnumber=37]{../src/main/java/vs/chat/server/listener/LoginListener.java}
        Anschließend wird der Client auf den neuesten Stand gebracht indem ein LoginSyncPacket an den Client gesendet wird. Dieses enthält die User-Id des aktuellen Nutzers, die anderen registrierten Nutzer und alle Chats, an dem der Client teilnimmt.
        \lstinputlisting[linerange={59-65}, firstnumber=59]{../src/main/java/vs/chat/server/listener/LoginListener.java}
    \item MessageListener\\
        Messages, die vom Client an einen Chat gesendet werden, werden von diesem Listener bearbeitet. Der Listener kümmert sich dabei auch um die Verteilung der Nachrichten an alle anderen Chatteilnehmer.
        \lstinputlisting[linerange={24-44}, firstnumber=24]{../src/main/java/vs/chat/server/listener/MessageListener.java}
    \item NodeSyncListener\\
        Wie in Fehlerbehandlung beschrieben müssen Nodes auf dem neuesten Stand gezogen werden, falls diese ausgefallen waren. Bei einem Reconnect wird ein NodeSyncPacket mit den aktuellen Informationen an die neu startende Node gesendet. Dieser Listener verarbeitet diese Pakete indem er prüft ob eine Änderung vorliegt und wenn ja diese übernimmt und broadcastet. 
        \lstinputlisting[linerange={16-34}, firstnumber=16]{../src/main/java/vs/chat/server/listener/NodeSyncListener.java}
        Theoretisch kann es sein, dass ein Nutzer sich anmeldet bevor die Node ihre Synchronisation abgeschlossen hat. Dieser Fehlerfall wird aber nicht weiter behandelt, da die Synchronisationszeit und die Ausfallwahrscheinlichkeit einer Node als zu gering eingestuft wird.
\end{itemize}

\subsection{Client}
API-Funktionen
\begin{itemize}
    \item Funktion - login\\
    Bei der Funktion login werden der Benutzername und das Passwort des Benutzers entgegengenommen. Diese werden als neues LoginPacket an den Server geschickt. Dort wird unter anderem überprüft, ob es sich um einen neuen Nutzer handelt (es wird ein Neuer angelegt) oder es ein bereits existierender Nutzer ist (Passwort wird überprüft). An dieser Stelle wartet der Client auf ein LoginSyncPacket (Login war erfolgreich). Danach werden die Attribute (userId, chats, contacts) abgespeichert. Für die Verschlüsselung wird auch noch die Keyfile geladen, in der die Schlüssel der eigenen Chats gespeichert sind. Sollte es noch keine Keyfile geben, wird eine neue erzeugt. Diese Datei ist ähnlich wie die Warehouse-Datei aufgebaut (hier wird über die Chat-Id, die Schlüssel geladen). Abschließend wird aus den beiden festgeschriebenen, öffentliche Primzahlen noch ein Schlüssel generiert, der später für den Diffie-Hellman-Key-Exchange notwendig ist.
    \lstinputlisting[linerange={119-149}, firstnumber=58]{../src/main/java/vs/chat/client/ClientApiImpl.java}
    \item Funktion - generatePrivateKey\\
    Mit dieser Funktion wird ein 128 Schlüssel aus den beiden öffentlichen Primzahlen n und g generiert. Der Schlüssel wird für die Verschlüsselung verwendet.
    \lstinputlisting[linerange={159-167}, firstnumber=95]{../src/main/java/vs/chat/client/ClientApiImpl.java}
    \item Funktion - exchangeKeys\\
    Um den Diffie-Hellman-Key-Exchange zu starten muss die \textit{exchangeKeys} Methode vom Client aufgerufen werden.
    Dabei erstellt die Methode das erste KeyExchangePacket, welches mit den erforderlichen Informationen
    ausgestattet wird. Nachdem das Paket gesendet wurde, haben die Teilnehmer zehn Sekunden Zeit, um den Schlüsseltausch durchzuführen.
    Wenn dies innerhalb der angegebenen Zeit nicht geschieht, wird von einem Fehler ausgegangen und die Error-Handling-Methode \textit{onTimeout} aufgerufen.
    \lstinputlisting[linerange={195-216}, firstnumber=154]{../src/main/java/vs/chat/client/ClientApiImpl.java}
    \item Funktion - createChat\\
    Für einen neuen Chat werden zwei Parameter benötigt: Der Name des Chats und eine Liste mit Usern. Um einen neuen Chat zu
    erstellen, wird ein CreateChatPacket mit dem Chatnamen und einem Array der User-Ids an den Server geschickt.
    \lstinputlisting[linerange={218-225}, firstnumber=154]{../src/main/java/vs/chat/client/ClientApiImpl.java}
    \item Funktion - sendMessage\\
    Hier wird die eigentliche Nachricht und eine Chat-Id entgegengenommen. Diese werden dann in ein MessagePacket an
    den Server geschickt.
    \lstinputlisting[linerange={227-233}, firstnumber=163]{../src/main/java/vs/chat/client/ClientApiImpl.java}
    \item Funktion - Verschlüsselung (encryptAES, decyptAES, setKey) \\
    Für die symmetrische Verschlüsselung der Nachrichten wird der AES-Algorithmus benutzt. Hierfür wird vor jedem Senden
    die Methode \textit{encryptAES} und nach jeder empfangener Nachricht \textit{decryptAES} aufgerufen.\\
    Für die Verschlüsselung wird zunächst der Schlüssel in das richtige Format (SecretKeySpec), durch die Funktion
    \textit{setKey} gebracht. Anschließend wird eine Instanz der Klasse Cipher erzeugt und mit dem Verschlüsselungsmodus und
    dem Key initialisiert. Abschließend wird der eigentlich Text verschlüsselt und in einem String zurückgegeben. \\
    Die Entschlüsselung ist fast identisch zur Verschlüsselung. Hier wird die Instanz in dem Entschlüsselungsmodus
    initialisiert.
    \lstinputlisting[linerange={235-255}, firstnumber=171]{../src/main/java/vs/chat/client/ClientApiImpl.java}
    \item Funktion - Keyfile (addKey, loadKey, deleteKey) \
    addKey\\
    Mit diesen Funktionen wird auf die Keyfile des Benutzers zugegriffen. Die Keyfile besteht aus PrivateKeyEntitys.
    Der Schlüssel ist die chatId.
    Wird ein neuer Key gespeichert, wird eine neue PrivateKeyEntity mit der chatId und dem Schlüssel in der keyfile
    abgespeichert.
    
    loadKey\\
    Müssen Nachrichten angezeigt werden, muss der Schlüssel aus der Datei für die chatId ausgelesen werden.
    
    deleteKey \\
    Sollte ein Chat irgendwann gelöscht werden, kann der die PrivateKeyEntity über die chatID entfernt werden. 
    
    deleteKey\\
    //TODO hier fehlt noch text + das listing ist verschoben
    \lstinputlisting[linerange={257-301}, firstnumber=209]{../src/main/java/vs/chat/client/ClientApiImpl.java}
    \item Funktion - exit \\
    Bei dieser Methode wird ein LogoutPacket an den Server geschickt. Der User wird hier noch nicht ausgeloggt!
    \lstinputlisting[linerange={312-316}, firstnumber=248]{../src/main/java/vs/chat/client/ClientApiImpl.java}
    \item PacketListener-Thread \\
    Dieser Thread läuft die gesamte Zeit im Hintergrund und nimmt alle (außer LoginSynPacket) Packete, die vom Server
    an den Client geschickt werden, auf und verarbeitet sie.
    Zuerst wird überprüft, ob es sich um ein BaseEntityBroadcastPacket handelt. Anschließend gibt es verschiedene Möglichkeiten: \\
    \begin{enumerate}
        \item Es wurde ein Chat erstellt, in dem der User enthalten ist. Hier wird der Key in der Keyfile abgespeichert
        und der Chat zu den Chats des Users hinzugefügt.
        \item Es wurde eine Nachricht an den User geschrieben. Hier wird der Schlüssel aus der Keyfile geladen und die
        Nachricht damit entschlüsselt.
        \item Es hat sich ein neuer User angemeldet. Dieser User muss den Usern des Clients hinzugefügt werden, damit auch
        Chats mit diesem erstellt werden können. (Diffie Hellman funktioniert nur, wenn alle Nutzer online sind.)
    \end{enumerate}
    \lstinputlisting[linerange={326-356}, firstnumber=262]{../src/main/java/vs/chat/client/ClientApiImpl.java}
    Handelt es sich um ein KeyExchangePacket, wird überprüft, ob der Schlüsseltausch schon fertig ist,
    falls ja erstellt der Initiator des Schlüsseltauschs den Chat, falls nicht werden die nächsten
    Teilschlüssel berechnet und weitergeschickt.\\
    \lstinputlisting[linerange={358-403}, firstnumber=294]{../src/main/java/vs/chat/client/ClientApiImpl.java}
    Wird ein GetMessagesResponsePacket erhalten, wurde an den Server zuvor die Anfrage gestellt, dass die Chat Historie
    geladen werden soll. Dieses Packet enthält alle Nachrichten, die in dem geladenen Chat bereits geschrieben wurden.
    Dafür wird zunächst der passende Schlüssel aus der Keyfile geladen und danach alle Nachrichten entschlüsselt.
    Diese Nachrichten können dann im Chat angezeigt werden.\\
    \lstinputlisting[linerange={405-419}, firstnumber=341]{../src/main/java/vs/chat/client/ClientApiImpl.java}
    Abschließend wird geprüft, ob es ein LogoutSuccessPacket ist. Dieses Paket wird vom Server geschickt, wenn der
    User die exit Methode aufgerufen hat und erfolgreich vom Server ausgeloggt wurde. Anschließend wird der Client
    heruntergefahren.\\
    \lstinputlisting[linerange={430-438}, firstnumber=356]{../src/main/java/vs/chat/client/ClientApiImpl.java}
Verwendung von Emojis: \\\\
	Wie bereits in der Doku erwähnt nutzen wir für die Darstellung von Emojis Unicode.
	\lstinputlisting[linerange={621-631}, firstnumber = 621]{../src/main/java/vs/chat/client/UI/ClientGUI.java}
	Die im String Array unicodeemoji erhaltenen Unicode Zeichen werden jeweils in einem JLabel hinzugefügt und auf ein Panel gesetzt. Jedes JLabel bekommt den selben Mouselistener zugewiesen, der als Übergabeparameter das entsprechende Unicodezeichen erhält. Im EmojiMouselistener wird mit der Funktion append() die JTextArea um das jeweils übergebene Unicodezeichen erweitert.
	\lstinputlisting[linerange={89-109}, firstnumber = 89]{../src/main/java/vs/chat/client/UI/ClientGUI.java}
	Durch das Erweitern oder Ändern des unicodeemoji Arrays können die Emoticons getauscht oder beliebig erweitert werden. Weiter Anpassungen sind nicht nötig.
\end{itemize}



    
\end{document}